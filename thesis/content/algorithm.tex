\chapter{Algorithm}
\label{chap:algorithm}
This thesis provides an implementation and improvement of the alignment algorithm proposed in \cite{hundt2020praktkium}. It shares the idea of finding segment pairs, or micro alignments, in the input sequences which are greedily aligned with Dialign. However in contrast to that approach, these micro alignments are found by searching for spaced word matches with multiple patterns, a process that is much faster than doing a pairwise alignment for all input sequence combinations. 
%\begin{itemize}
%	\item describe how micro alignments, especially multi dim are found?
%	\item what scoring function should be used and why?
%	\item TODO maybe have a look at overlap weight again
%	\item describe version of algorithm utilizing eq classes and property described in \ref{fig:unnecessary-cmp} -> should this include data structures and memory evincemanagement?
%	\item maybe first dp algorithm without unnecessary cmps (like in Fi. 4 of \cite{abdeddaim1997incremental}) and the formulate with eq classes (which wasn't done in paper) 
%	\item highlight how this differs from most alignment algorithms in that it is completely greedy
%\end{itemize}

% TODO the algorithms input and output data needs
% to be cleaned up

\begin{algorithm}[h]
	\DontPrintSemicolon
	\KwData{Sequences $S = {S_1, ..., S_N}$}
	\KwData{Pattern set $P = {P_1, ..., P_M}$}
	\KwResult{Partial Alignment $A$ }
	\SetKwFunction{sort}{sort\_by\_score\_descending}
	\SetKwFunction{incons}{is\_inconsistent}
	\SetKwFunction{findspam}{find\_spaced\_word\_matches}
	\SetKwFunction{addsite}{add\_site\_pair}
	\SetKwFunction{notaligned}{not\_aligned}
	\SetKwFunction{insertgaps}{insert\_gaps}
	
	
	$A \leftarrow$ $\{\}$\;
	
	micro\_alignments $\leftarrow$ \findspam{$S$, $P$}\;
	micro\_alignments.\sort{}\;
	
	\ForEach{ma in micro\_alignments}{
		\If{\incons{ma, $A$}}{
			continue\;
		}
		\ForEach{site\_pair in ma}{
		\If{$A$.\notaligned{site\_pair}} {
				$A$.\addsite{site\_pair}\;	
			}
		}
	}	
	\insertgaps{$S$, $A$}\;
	\Return $S$\;
	
	\caption{\bf{align($S$, $P$)}}
	\label{alg:align}
\end{algorithm}

\section{Finding spaced word matches}

TODO: describe find\_spaced\_word\_matches

\section{Incrementally maintaining a transitive closure}

The, unfortunately incomplete, description of the \textit{EdgeAddition} algorithm in \cite{abdeddaim2000speeding} lead to the proposal of the algorithm \ref{alg:add-site-pair} in the preliminary work \cite{hundt2020praktkium}; due to the incomplete understanding it is computationally much more expensive than necessary since for every site pair that is added, many sites are considered whose transitivity frontiers can not be influenced.

\begin{algorithm}[h]
	\DontPrintSemicolon
	\KwData{Consistent site pair $a,b \in X$ to align}
	\KwData{Partial Alignment $A$}
	\KwResult{Partial Alignment $A' = A \cup \{(a, b)\}$ }

	\SetKwFunction{min}{min}
	\SetKwFunction{max}{max}
	
	\tcp{Clone the old pred and succ values}
	$pred \leftarrow$ $pred_A$\;
	$succ \leftarrow$ $succ_A$\;
	
	\tcp{Update the successor frontier}
	\ForEach {$x \in X$} {
		\For {$i \leftarrow 1$ \KwTo $N$} {
			\If{$ x \preceq_{A} a$}{
				$succ_A[x, i]$ $\leftarrow$ \min{$succ[x, i], succ[b, i]$}
			}
			\ElseIf{$ x \preceq_{A_i} b$} {
				$succ_A[x, i]$ $\leftarrow$ \min{$succ[x, i], succ[a, i]$}
			}
			\Else{
				$succ_A[x, i] \leftarrow succ[x, i]$
			}
		}
	}
	
	\tcp{Update the predeccessor frontier}
	\ForEach {$x \in X$} {
		\For {$i \leftarrow 1$ \KwTo $N$} {
			\If{$ x \succeq_{A} a$}{
				$pred_A[x, i]$ $\leftarrow$ \max {$pred[x, i], pred[b, i]$}
			}
			\ElseIf{$ x \succeq_{A_i} b$} {
				$pred_A[x, i]$ $\leftarrow$ \max{$pred[x, i], pred[a, i]$}
			}
			\Else{
				$pred_A[x, i] \leftarrow pred[x, i]$
			}
		}
	}
	\caption{add\_site\_pair(x, y) as proposed in \cite{hundt2020praktkium}}
	\label{alg:add-site-pair}
\end{algorithm}

Algorithm \ref{alg:add-site-pair-revised} represents a considerably improved version of \ref{alg:add-site-pair}. It is based on, and improves, the version of \textit{EdgeAddition} contained in Dialign2.2. The most significant improvement over \ref{alg:edge-addtition} stems from the fact that the transitivity frontiers for sites, which are aligned, coincide, as pointed out in \cite{abdeddaim2000speeding}. In other words, given $x \in X$ and an alignment $A$ the following holds $\forall x' \in [x]_A, \forall i \in {1, ..., N}$:
\begin{align*}
 pred_A[x'][i] &= pred_A[x][i] \text{ and}\\
 succ_A[x'][i] &= succ_A[x][i]
\end{align*}
This allows faster updates and a more compact representation for the transitivity frontiers, since only one value per equivalence class needs to be kept in memory. For those sites $x$ which are not yet aligned, $|[x]_A|$ is $1$, the transitivity frontiers are equal to its nearest aligned site.

\begin{figure}[ht]
	\tikzset{SeqNode/.style={circle, draw, fill=black, inner sep=0pt, minimum width=4pt}}
	\centering
	\begin{tikzpicture}[thick]
	
	\draw[-latex]
	(0, 0) node [] {$S_1$}
	(1,0) -- (2,0);
	\foreach \x in {2, ..., 6}
	{
		\draw[-latex](\x,0) node[SeqNode] {} -- (\x+1,0);
	}
	\draw[-latex]
	(0, 1) node [] {$S_2$}
	(1, 1) -- (2,1);
	\foreach \x in {2,..., 6}
	{
		\draw[-latex](\x, 1) node[SeqNode] {} -- (\x+1,1);
	}
	\draw[-latex]
	(0, 2) node [] {$S_3$}
	(1, 2) -- (2,2);
	\foreach \x in {2,..., 6}
	{
		\draw[-latex](\x, 2) node[SeqNode] {} -- (\x+1,2);
	}
	\draw[-latex]
	(0, 3) node [] {$S_4$}
	(1, 3) -- (2,3);
	\foreach \x in {2,..., 6}
	{
		\draw[-latex](\x, 3) node[SeqNode] {} -- (\x+1,3);
	}
	\draw
	(3, 0)+(-0.2, 0.2) node [] {a}
	(4, 1)+(-0.2, 0.2) node [] {b}
	(4, 2)+(-0.2, 0.2) node [] {c}
	(5, 2)+(-0, -0.3) node [] {[3, 4]};
	\draw[-, line width=1.2] (3,0) -- (4,1);
	\draw[-, line width=1.2] (4,1) -- (4,2);
	\draw[-, line width=1.2] (5,2) -- (6,3);
	\end{tikzpicture}
	\caption{Since $a, b, c$ are aligned, their shared successor frontier towards sequence $3$ is at position $4$.}
	\label{fig:eq-classes}
\end{figure}


\begin{algorithm}[h]
	\DontPrintSemicolon
	\KwData{Consistent site pair $x,y \in X$ to align}
	\KwData{Index $nn$ of alignment set that was merged from alignment sets of $x$ and $y$}
	\KwData{Successor and predecessor frontiers $succ, pred$}
	\KwData{$pred_x$ and $pred_y$ are predecessor frontiers of x and y respectevily}
	\KwData{$alig\_set$ matrix, mapping an alignment set and sequence to a position}
	\KwData{$pred\_alig\_set\_pos$ mapping a sequence and position to the index of the next predecessor alignment set of that site, $succ\_alig\_set\_pos$ equivalent for successor alignment set}
	\KwResult{Updated tranisitivity frontiers $succ$ and $pred$}
	\SetKwFunction{push}{push}
	
	\tcp{Calculate updates to successor frontier}
	$succ\_frontier\_ops \leftarrow [\ ]$
	
	\For {$i \leftarrow 1$ \KwTo $N$} {
		\If{$pred_x[i] == pred_y[i]$}{
			\bf{continue}\;
		}
		\For{$j \leftarrow 1$ \KwTo $N$}{
			$k \leftarrow pred[nn,\ i]$\;
			\If{$k > 0$ \bf{and} $k == alig\_set[nn,\ i]$}{
				$k \leftarrow pred\_alig\_set\_pos[i,\ k]$\;	
			}
			\While{$k > 0$}{
				$n \leftarrow alig\_set\_nbr[i,\ k]$\;
				\If{$succ[n,\ j] > succ[nn,\ j]$}{ \label{alg:add-site-pair-revised_data-dep}
					$succ\_frontier\_ops$.\push($[n,\ j,\ succ[nn,\ j]]$)\;
					$k \leftarrow pred\_alig\_set\_pos[i,\ k]$\;
				} 
				\Else{
					\bf{break}\;
				}
			}
		}
	}
	\tcp{Calculate updates to predecessor frontier}
	$pred\_frontier\_ops \leftarrow [\ ]$
	
	\For {$i \leftarrow 1$ \KwTo $N$} {
		\If{$succ_x[i] == succ_y[i]$}{
			\bf{continue}\;
		}
		\For{$j \leftarrow 1$ \KwTo $N$}{
			$k \leftarrow succ[nn,\ i]$\;
			\If{$k == alig\_set[nn,\ i]$}{
				$k \leftarrow succ\_alig\_set\_pos[i,\ k]$\;	
			}
			\While{$k > 0$}{
				$n \leftarrow alig\_set\_nbr[i,\ k]$\;
				\If{$pred[n,\ j] < pred[nn,\ j]$}{
					$succ\_frontier\_ops$.\push($[n,\ j,\ pred[nn,\ j]]$)\;
					$k \leftarrow succ\_alig\_set\_pos[i,\ k]$\;
				} 
				\Else{
					\bf{break}\;
				}
			}
		}
	}
	
	\ForEach{$[n,\ j,\ new\_front] \in succ\_frontier\_ops$}{
		$succ[n,\ j] \leftarrow new\_front$\;
	}
	\ForEach{$[n,\ j,\ new\_front] \in pred\_frontier\_ops$}{
		$pred[n,\ j] \leftarrow new\_front$\;
	}
	
	\caption{revised add\_site\_pair(x, y) based on GABIOS-LIB implementation in Dialign2.2 \cite{abdeddaim2000speeding}}
	\label{alg:add-site-pair-revised}
\end{algorithm}

\section{Inserting gaps to produce an alignment from transitivity closure}
Once the list of spaced word matches that are found in the sequences $S$ is filtered and added to the transitive closure, the information about the equivalence classes of aligned sites can be utilized to efficiently insert gaps into the input sequences such that those sites belonging to the same class are in the same column of the resulting alignment.\\
This can be achieved by using the quasi-order $\preceq_{A}$, originally defined on the set of sites $X$, as a partial order on the alignment equivalence classes $[x]_A$. This is allowed by the simple observation that $\forall x, y \in X, x' \in [x]_A \text{ and } y' \in [y]_A$, $x' \preceq_A y'$ if and only if $x \preceq_A y$ which leads to the definition $[x]_A \preceq_A [y]_A \defeq x \preceq_A y$. Anti-symmetry follows trivially and thus $\preceq_A$ partially orders the alignment equivalence classes.\\
Since the revised \textit{EdgeAddition} algorithm \ref{alg:add-site-pair-revised} already stores information about equivalence classes, it is easy to collect a partially ordered list of equivalence classes $E$. Using this sorted list, the sequences can be aligned by taking the first class, setting the positions of the aligned sites to the maximum position in the class and then shifting the sites of the eq classes coming after the aligned one by the number of gaps that were added to the sequences.  
